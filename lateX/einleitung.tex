\chapter{Einleitung}
\label{cha:Einleitung}

\section{Überblick}
In der modernen Softwareentwicklung geht es in erster
Linie um das Umsetzen funktionaler sowie nicht funktionaler
Anforderungen und um die Behebung von Fehlern welche bei der Umsetzung dieser Anforderungen häufig auftreten. Ein wichtiger Punkt dabei ist, den Code so zu gestalten, dass er nicht nur vom Programmierer, der ihn geschrieben hat, gelesen werden kann, sondern auch von anderen Programmieren und dies auch möglichst noch nach mehreren Monaten. Um dies zu erreichen, müssen gewisse Grundsätze angewendet werden. In den letzten Jahren hat sich zu diesem Thema eine Strömung ergeben welche sich als Clean Code Development (im folgenden nur noch als CCD abgekürzt) bezeichnet. Geprägt wurde diese Bewegung in erster Linie von Robert. C. Martin und seinem Bestseller Clean Coder. 

\section{Gliederung}
Die Arbeit ist in drei größere Teile gegliedert. Der erste Teil beschäftigt sich in erster Linie mit der Erklärung des CCD Begriffes und des ........................ 

\SuperPar Im zweiten Teil werden die einzelnen Regeln und Grundsätze anhand von Beispielen erläutert wie sie den Code verbessern bzw. welche Änderungen an Codestücken vorgenommen werden müssen, um diese Regeln und Grundsätze zu entsprechen.

\SuperPar 
Der dritte Teil beschäftigt sich mit dem Analysieren von Opensource Frameworks hinsichtlich der CCD Kriterien und ihrer Erfüllung. Es sollten dabei Codestücke gezeigt werden, welche die Kriterien erfüllen, sowie Codestücke die diese nicht erfüllen und dann abgeändert werden. 

