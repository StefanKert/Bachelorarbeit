\chapter{Einleitung}
\label{cha:Einleitung}

\section{Problemstellung}


\section{Motivation}
In der modernen Softwareentwicklung geht es in erster Linie um das Umsetzen funktionaler sowie nicht funktionaler
Anforderungen und um die Behebung von Fehlern welche bei der Umsetzung dieser Anforderungen häufig auftreten. Ein wichtiger Punkt dabei ist, den Code so zu gestalten, dass er nicht nur vom Programmierer, der ihn geschrieben hat, gelesen werden kann, sondern auch von anderen Programmieren und dies auch möglichst noch nach mehreren Monaten. Um dies zu erreichen, müssen gewisse Grundsätze angewendet werden. In den letzten Jahren hat sich zu diesem Thema eine Strömung ergeben welche sich als Clean Code Development (im folgenden nur noch als CCD abgekürzt) bezeichnet. Geprägt wurde diese Bewegung in erster Linie von Robert. C. Martin und seinem Bestseller Clean Coder. Umso genauer man darauf achtet, den Code beim Schreiben lesbar zu gestalten, umso leichter wird es diesen auch noch nach einiger Zeit wieder zu Lesen. Bei schlecht geschrieben Code kann es sein, dass man bereits nach einigen Tagen nicht mehr genau weiß, was man damit bezwecken woltle. 

\section{Lösungsvorschlag}