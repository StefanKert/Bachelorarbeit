\chapter{Abstract}

\begin{english} 
In software development, it is extremely important to write programs so that they are easy understandable and maintainable, even after many iterations. To make this possible, it is very important to assure that the source code is already kind of \textit{clean} when it is written. For this purpose there have been found a few best practices to define rules for writing \textit{clean Code}. One of the most significant work in this field is done by Robert C. Martin with his book \textit{Clean Code} \cite{Martin2008}. In this book he describes his approach to write clean code, the benefits of writing clean code and many simple rules that are really helpful for simply writing better code.

This thesis takes \textit{Clean Code} \cite{Martin2008} as basic to address the principles of Clean Code Development and explain them. The thesis is splitted into severel sections which deal with the basics, the application and the everyday use of CCD. In the basic section an impression should be conveyed what CCD is and the benefits of its use, as well as the origin and the pioneer of CCD. For this purpose it is explained what clean code means and also what clean code means to some of the most famous programmers such as Grady Booch and Michael Feathers. These opinions, that Robert C. Martin collected for his book Clean Code \cite{Martin2008}, are used then to explain the basics of clean code. Further to the basics of clean code there are mentioned some possiblities to check some of the rules automatically or manually and also some tooles that make it easier to apply these rules.

In the second part some of the rules which are mentioned in the first part are exemplified. Every example also contains a description, which problems are resolved by applying these rules.

The third part will deal primarily with a personal opinion about CCD and the use in everyday life.
\end{english}