\chapter{Abstract}

\begin{english} 
In software development, it is extremely important to write code so that it is easy understandable and maintainable, even after many iterations. To make this possible, it is very important to assure that the source code is already kind of \textit{clean} when it is written. In the last few years, there have been found a range of best practices and rules for making it easier to write \textit{clean code}. One of the most significant works in this field is done by Robert C. Martin with his book \textit{Clean Code}. In this book he describes his approach to write clean code, the benefits of writing clean code and many simple rules that are really helpful for simply writing better code.

This thesis takes \textit{Clean Code} as a base to address the principles of Clean Code Development and to describe these principles. The thesis is splitted into severel sections which deal with the basics, the application and the continuous use of CCD. In the basic section an impression should be conveyed what CCD is and also what clean code means to some of the most known programmers such as Grady Booch and Michael Feathers as well as the benefits of its use. Another section is about the origin and the pioneer of CCD. 

In the second part some of the CCD rules are exemplified and it is described, which problems are resolved by applying these rules. Most of the examples used hail from various Open Source frameworks.

\end{english}