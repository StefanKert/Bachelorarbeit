\section{Nutzen moderner Werkzeuge zur Softwareentwicklung}
Etwas das Robert C. Martin sehr oft in seinem Buch kritisiert, ist die Tatsache, dass viele Entwickler die modernen Werkzeuge, die ihnen zur Verfügung stehen nicht nutzen und daher unnötige Arbeiten immer wieder erledigen. Dabei wurde bereits im Abschnitt \ref{cha:CommentedCode} erläutert, welche Probleme entstehen, wenn die vorhandenen Werkzeuge wie z.B. Versionsverwaltungssysteme nicht korrekt eingesetzt werden.

\subsection{Testen, Erzeugen und Verteilen von Software}
In der modernen Softwareentwicklung spricht man oft von \textit{Continous Delivery}. Dabei geht es um die azyklische Auslieferung von Software. Es wird dabei eine Pipeline eingerichtet, welche die einzelnen Aufgaben, welche für die Auslieferung von Software nötig sind ausführt. In dem Buch \textit{Continuous Delivery} von Eberhard Wolff erläutert der Autor seine Gedanken zum automatisieren des Prozesses vom Erstellen der Software, über das Testen bis zur Auslieferung dieser. Dies sollte für das Team möglichst einfach möglich sein und einmalig eingerichtet, für jeden zugänglich sein, sodass man möglichst schnell eine Rückmeldung erhält, ob die aktuellen Änderungen Probleme verursacht haben. Dies sollte für das Team über eine sogenannte \textit{Continous Integration} Komponente möglich sein.

\SuperPar Der automatische Prozess des Erzeugens und des Testens sollte jedoch nicht nur über die \textit{Continous Integration} Komponente ermöglicht werden. Es sollte für jeden Programmierer durch wenige Schritte möglich sein, die Software zu Erzeugen und die Tests auch auszuführen. Dies sollte möglichst durch einen einzigen Befehl, durch einen einzigen Klick, oder wenn möglich sogar automatisiert erledigt werden. Es sollten dafür keine komplizierten Operationen nötig sein, da dies zu einer Duplizierung der Arbeit führt und immer wieder Zeit benötigt. Mit modernen Entwicklungsumgebungen wie Visual Studio oder Eclipse sind meist schon Werkzeuge integriert, welche das Erstellen und das Testen von Software durch eine einfache Tastenkombination oder einen einzelnen Klick ermöglichen. Diese Features sollten daher auch verwendet werden.

\subsection{Werkzeuge für das Refactoring}
Nahezu alle modernen Entwicklungsumgebung bieten Möglichkeiten für ein effizientes und einfaches Ändern von Namen. Dabei übernimmt die Entwicklungsumgebung alle Aufgaben, die für eine korrekte Umbenennung nötig sind. Alle Referenzen werden umbenannt, teilweise werden sogar in Kommentaren Namen geändert die auf diesen Namen verweisen. Durch dieses Feature ist es nicht mehr nötig, eine Umbenennung auf manuelle Art durchzuführen, da dies meist zu zahlreichen Fehlern führt. 
