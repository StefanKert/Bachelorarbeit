\chapter{Einleitung}
\label{cha:Einleitung}

\section{Problemstellung}
\label{cha:Problemstellung}
Ein sehr zentraler Aspekt der Softwareentwicklung ist die Notwendigkeit, Software ständig zu ändern. Dies kann aus verschiedenen Gründen nötig werden. Ein Kunde möchte ein neues Feature implementiert haben, bei einem anderen Kunden ist ein Fehler aufgetreten der behoben werden sollte. Ein weiterer Kunde beschwert sich über die sehr schlechte Performance einer Funktionalität. Alle diese Fälle haben eines gemeinsam; sie erfordern eine Änderung der bestehenden Software. Wenn man in der glücklichen Lage ist ein neues Produkt zu entwickeln ist es oft noch sehr leicht weitere Funktionalität hinzuzufügen, aber Neuentwicklungen, sogenannte \textit{ Greenfield Applications} sind sehr selten der Fall. Meistens muss die Funktionalität zu einer bereits vorhandenen, meist schon länger entwickelten, Codebasis hinzugefügt werden. In vielen Fällen kommen jetzt zahlreiche Probleme zum Tragen: Es wurde beim ersten Entwurf der Software nicht daran gedacht, dass diese Feature benötigt wird, daher ist es durch die gewählte Architektur nur sehr schwer möglich die gewünschte Funktionalität hinzuzufügen. Meist wird irgendein Workaround gefunden, dass die Funktionalität trotzdem realisiert werden kann, jedoch führt dies zu einer weiteren Verunstaltung des Codes und er wird dadurch noch schwerer zu lesen und zu warten. 

Diese verschlechterte Lesbarkeit und Wartbarkeit führt dazu, dass Änderungen sehr viel Zeit benötigen und nur sehr ungern vorgenommen werden. Im schlimmsten Fall ist irgendwann der Punkt erreicht, an dem die Software nicht mehr weiterentwickelt werden kann, da jede Änderung unvorhersebare Seiteneffekte haben könnte. Dies führt schlussendlich wohl oder übel zum Tod der Software, den Software die sich nicht weiterentwickeln lässt wird früher oder später von besserer Software abgelöst. 

\subsection{Theorie der zerbrochenen Fenster}
Die Autoren James Q. Wilson und George L.Kelling haben in Ihrem Artikel \cite{Wilson2016} ihre Theorie der zerbrochenen Fenster erläutert. Bei dieser Theorie geht es darum, welche Folgen ein zerbrochenes Fenster in einem Gebäude nach sich zieht insofern dieses nicht repariert wird. Die Autoren erläutern dabei, dass durch die Tatsache, dass es aussieht als ob sich niemand um den Zustand dieses Gebäudes kümmern würde, weitere Zerstörung nach sich ziehen wird. Robert C. Martin, der Autor von \cite{Martin2008} verwendet diese Theorie und bringt sie in Verbindung mit der Softwareentwicklung. Dabei beschreibt er einen sehr ähnlichen Fall wie der eines zerbrochenen Fensters. Bei einem Programmabschnitt, sei es eine Klasse, eine Methode oder ähnliches, die bereits unsauber programmiert und schlecht lesbar ist, wird bei zukünftigen Änderungen auch nicht darauf geachtet werden den Code sauber zu gestalten. Es werden immer mehr unsaubere Codeteile hinzugefügt, was schlussendlich von unsauberen Code zu nicht mehr wartbarem Code führen kann. Dieses Problem existiert in allen Teilen der Programmierung. Bei Projekten, bei denen nicht auf einen sauber gestalteten Code geachtet wurde, wird in Zukunft auch nur selten darauf geachtet.

\section{Lösungsvorschlag}
Um diese Probleme zu vermeiden, sollten Regeln und Grundsätze gefunden werden, wie gut lesbarer bzw. wartbarer Code gestaltet werden kann. Dabei ist es sehr wichtig darauf zu achten, dass der Code meist nicht nur von einer Person bearbeitet und gelesen wird, sondern von vielen verschiedenen Personen. Wenn darauf geachtet wird denn Code so zu gestalten, dass er verständlich, gut lesbar und gut strukturiert ist, dann ist es auch nach einigen Wochen, Monaten und Jahren noch gut möglich zu verstehen, was der Zweck der betrachteten Codestelle ist. In den letzten Jahren hat sich zu diesem Thema eine Strömung ergeben welche sich als Clean Code Development (im folgenden nur noch CCD) bezeichnet. Geprägt wurde diese Bewegung in erster Linie von Robert. C. Martin und seinem Buch \cite{Martin2008}. Er beschreibt dabei, dass umso genauer beim Schreiben des Codes darauf geachtet wurde, diesen lesbar zu gestalten, desto leichter wird es für zukünftige Leser des Codes diesen zu verstehen und auch zu ändern. 

