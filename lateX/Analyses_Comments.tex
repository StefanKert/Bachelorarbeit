\section{Kommentare}
In der Softwareentwicklung sind Kommentare ein Mittel um dem Leser verschiedene Dinge zu vermitteln.
Einerseits können Gründe vermittelt werden, wieso etwas so implementiert wurde, oder was sich der 
Autor dabei gedacht hat. Auch rechtliche Kommentare, meist gibt es diese bei Opensource Projekten,
sind ein sehr häufige Form der Kommentare. Für Bibliotheken gibt es weiters die Möglichkeit, Beschreibungen
für Parameter, Rückgabewerte und allgemein den Zweck von Methoden oder Klassen zu erläutern. 

Ein sehr problematisches Thema beim Kommentieren von Quellcode ist die Tatsache, dass oft darauf vergessen wird
bei einem Refactoring diese auch zu ändern. 

\subsection{Redundante Kommentare}
\begin{itemize}
	\item Projekt: \textit{Log4Net}
	\item Programmiersprache: \textit{C\#}
	\item Betreffende Klasse: \textit{ConfigurationChangedEventArgs}
	\item Betreffender Namespace: \textit{log4net.Repository}
\end{itemize}

Redundante Kommentare findet man sehr häufig in den verschiedensten Projekten. Oft werden für Methoden Kommentare geschrieben, welche nicht mehr als den Namen der Methode beinhalten. Dies führt natürlich zu keiner Verbesserung der Lesbarkeit und führt im Weiteren auch zu einer Aufblähung des Codes. Sehr häufig tritt dieses Problem auch bei Vorgeschriebenen Kommentaren auf. bei denen es in Coding Conventions vorgeschrieben ist, dass jede Methode einen Header zur Dokumentation
bekommen sollte. Dabei kommt es häufig zu redundanten Kommentaren, welche keinen wirklichen Mehrwert bringen. Robert C. Martin schreibt hierzu (Clean Code, Seite 93 - 96), dass solche Regeln meist zu Verwirrung, Lügen und einer allgemeinen Unordnung führen. 

Wenn man nun das in Listing \ref{lst:RedundantComment} stehende Beispiel betrachtet, fällt einen sofort der Kommentar auf, welcher keine richtigen Mehrwert
für den Leser bringt. Wie Robert C. Martin erwähnt, führt dieser nur zu einer Unordnung und kann im schlimmsten Fall sogar zu einer fälschlichen Information führen,
falls der Parameter umbenannt wird und der Kommentar dafür nicht angepasst wird. Auf Grund dieser Tatsache, sollte man laut Robert C. Martin auch auf solche Regeln verzichten, da diese eben genau zu den genannten Probleme führen.

\begin{lstlisting}[language={[Sharp]C}, caption=Beispiele für überflüssige Kommentare, label=lst:RedundantComment]
/// <summary>
/// 
/// </summary>
/// <param name="configurationMessages"></param>
public ConfigurationChangedEventArgs(ICollection configurationMessages)
{
		this.configurationMessages = configurationMessages;
}
\end{lstlisting}

\subsection{Auskommentierter Code}
\begin{itemize}
	\item Betreffende Klasse: \textit{BindHelper}
	\item Betreffendes Paket: \textit{org.hibernate.cfg}
\end{itemize}


\SuperPar Code der nicht mehr benötigt wird, wird häufig einfach auskommentiert und bleibt danach über lange Zeit im Quellcode bestehen. Dies wäre jedoch bei den modernen Versionsverwaltungssystemen gar nicht mehr notwendig, da diese eine genaue Auflistung der gelöschten, geänderten oder hinzugefügten Abschnitte anbieten. Es ist mit diesen auch leicht möglich Abschnitte, die man gelöscht hat, wieder aufzufinden, sowie diese wiederherzustellen. Code der nicht mehr benötigt wird sollte daher einfach gelöscht werden und mit einer vernünftigen Commit Message versehen werden. Im Hibernate Framework ist eine Klasse die einen solchen auskommentierten Codeabschnitt enthält die \textit{BindHelper} Klasse. Eine sehr problematische Stelle befindet sich in dieser Klasse in Zeile 421. Dort gibt es den in Listing \ref{lst:CommentedCode} beschriebenen Codeabschnitt.

\begin{lstlisting}[language=Java, caption=Beispiele für die Verwendung von \textit{GetByPredicate}, label=lst:CommentedCode]
/*FIXME cannot use subproperties becasue the caller needs top level properties
//if (property.isComposite()) {
//	Iterator subProperties = ((Component)property.getValue()).getPropertyIterator();
// 	while (subProperties.hasNext()) {
//  	matchColumnsByProperty(((Property)subProperties.next()), columnsToProperty);
// 	}
}*/ 
\end{lstlisting}

Der Kommentar deutet darauf hin, dass es in diesem Codeabschnitt einen Fehler gibt der behoben werden müsste. Anstatt diesen Fehler zu beheben wurde der Code einfach auskommentiert und nach einigen Wochen weiß niemand mehr, dass es diesen Fehler gibt. Hier sollte entweder in einem Issue Tracking System genau mitdokumentiert werden, wo der Fehler auftritt und Möglichkeiten diesen zu beheben, oder den Fehler direkt zu beheben. Diesen einfach stehen zu lassen und die fehlerhafte Codestelle auszukommentieren ist dabei wohl der schlechteste Weg, da so der Fehler nicht mehr auftreten wird und er somit vergessen wird, wodurch sich vermutlich weitere Probleme ergeben.