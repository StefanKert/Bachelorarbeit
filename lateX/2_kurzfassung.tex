\chapter{Kurzfassung}

Für die Softwareentwicklung ist es von enormer Bedeutung, Programme so zu schreiben, dass diese auch nach vielen Iterationen noch gut zu warten und gut verständlich sind. Um dies zu ermöglichen ist es von sehr großer Bedeutung, dass bereits beim Schreiben des Quelltextes darauf geachtet wird, dass dieser \textit{clean} ist. Dazu haben sich in den letzten Jahren verschiedene Vorgehensweisen etabliert, um Regeln für das Schreiben von \textit{Clean Code} zu definieren. Die wohl bedeutendste Arbeit auf diesem Gebiet hat Robert C. Martin mit seinem Buch Clean Code \cite{Martin2008} geleistet. Er beschreibt in diesem Buch, was für ihn Clean Code bedeutet, welche Vorteile es hat Clean Code zu schreiben und wie man auf sehr einfache Art und Weise besseren Quelltext produzieren kann. 

Diese Bachelorarbeit nimmt \textit{Clean Code} \cite{Martin2008} zur Basis, um die Grundsätze von Clean Code Development aufzuarbeiten und diese zu erläutern. Dabei wird die Arbeit in mehrere Abschnitte gegliedert die sich unter anderem mit den Grundlagen, der Anwendung und dem alltäglichen Gebrauch von CCD beschäftigen. In den Grundlagen sollte ein Eindruck vermittelt werden, wofür CCD steht und welche Vorteile dessen Anwendung bringt, sowie der Ursprung und die Vorreiter von CCD. Dabei wird unter anderem darauf eingegangen, worum es sich bei Clean Code handelt und was Clean Code für einige der bekanntesten Programmierer wie z.B. Grady Booch und Michael Feathers bedeutet. Dabei werden diese Meinungen, welche Robert C. Martin für sein Buch Clean Code \cite{Martin2008} eingeholt hat verwendet um die Grundlagen von Clean Code zu erläutern. In den Grundlagen werden weiters Möglichkeiten aufgezeigt, wie einige der CCD Regeln automatisch oder manuell überprüft werden, sowie Tools, welche ein leichteres Anwenden dieser Regeln ermöglicht.

Im zweiten Teil der Arbeit werden einige der Regeln, welche im ersten Teil erwähnt werden, beispielhaft gezeigt und erläutert, welche Probleme durch Anwendung dieser Regeln behoben werden. 

Der dritte Teil wird sich in erster Linie mit einer persönlichen Meinung zu CCD und dem Gebrauch im Alltag beschäftigen.