\chapter{Kurzfassung}

Für die Softwareentwicklung ist es von ernormer Bedeutung, Programme so zu schreiben, dass diese auch nach vielen Iterationen noch gut zu warten und gut verständlich sind. Um dies zu ermöglichen ist es von sehr großer Bedeutung, dass bereits beim Schreiben des Quelltextes darauf geachtet wird, dass dieser \textit{clean} ist. Dazu haben sich in den letzten Jahren verschiedene Vorgehensweisen etabliert, um Regeln für das Schreiben von \textit{Clean Code} zu ermöglichen. Die wohl bedeutenste Arbeit auf diesem Gebiet hat Robert C. Martin mit seinem Buch Clean Code \ref{Martin2008} geleistet. Er beschreibt in diesem Buch, was für ihn Clean Code bedeutet, welche Vorteile es hat Clean Code zu schreiben und wie man auf sehr einfache Art und Weise besseren Quelltext produzieren kann. 

Diese Bachelorarbeit nimmt \textit{Clean Code} zur Basis, um die Grundsätze Ziel dieser Arbeit ist es, die Grundsätze des Clean Code Development zusammenzufassen und einige der von Robert C. martin definierten Regeln an konkreten Beispielen anzuwenden.