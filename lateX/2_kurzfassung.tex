\chapter{Kurzfassung}

Für die Softwareentwicklung ist es von enormer Bedeutung, Quellcode so zu schreiben, dass dieser auch nach vielen Iterationen noch gut zu warten und gut verständlich sind. Um dies zu ermöglichen ist es nötig, dass bereits beim Schreiben des Quellcodes darauf geachtet wird, dass dieser \textit{clean} ist. In den letzten Jahren haben sich dazu zahlreiche verschiedene Vorgehensweisen und Regeln etabliert, die das Schreiben von \textit{Clean Code} erleichtern sollten. Die wohl bedeutendste Arbeit auf diesem Gebiet hat Robert C. Martin mit seinem Buch Clean Code geleistet. Er beschreibt in diesem Buch, was Clean Code für ihn bedeutet, den Nutzen und die Vorteile von Clean Code und wie es auf sehr einfache Art und Weise möglich ist, besser verständlichen und wartbaren Quellcode zu schreiben.

Diese Bachelorarbeit nimmt \textit{Clean Code} zur Basis, um die Grundsätze von Clean Code Development aufzuarbeiten und diese zu erläutern. Dabei wird die Arbeit in mehrere Abschnitte gegliedert, die sich unter anderem mit den Grundlagen, der Anwendung und dem kontinuierlichen Gebrauch von CCD beschäftigen. In den Grundlagen sollte ein Eindruck vermittelt werden, worum es sich bei Clean Code handelt und was Clean Code für einige der bekanntesten Programmierer wie z.B. Grady Booch und Michael Feathers bedeutet. Dabei wird besonders auf die Vorteile eingegangen, die sich durch die Anwendung von CCD ergeben. Es wird außerdem auf den Ursprung und etwaige Vorreiter von CCD eingegangen.

Im zweiten Teil der Arbeit werden einige der CCD Regeln beispielhaft dargestellt und gezeigt welche Probleme durch Anwendung dieser Regeln behoben werden. Die meisten der verwendeten Beispiele stammen dabei aus verschiedenen Open Source Frameworks.