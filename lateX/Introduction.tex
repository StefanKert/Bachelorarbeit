\chapter{Einleitung}
\label{cha:Einleitung}

\section{Problemstellung}
\SuperPar In der Softwareentwicklung gibt es einen sehr zentralen Aspekt der sich nahezu über alle Elemente des Prozesses der Softwareentwicklung erstreckt oder Einfluss auf diese hat. Es geht in erster Linie immer darum, ein Produkt weiterzuentwickeln, bestehende Fehler zu beheben, oder neue Funktionalität hinzuzufügen. Dieser Änderungsprozess bezieht sich jedoch nicht nur auf die Software selbst, sondern auch auf die gesamte Infrastruktur, sowie die Menschen die mit der Software arbeiten, diese entwickeln, oder planen. Die Änderungen, welche in dem Bereich Infrastruktur stattfinden sind sehr tiefgreifend. Ein neues Betriebssystem für das die Software angepasst werden muss wäre ein Beispiel für solch eine Änderung. Häufig kommt es auch zu einer Änderung der Tools mit denen der Programmierer oder die Programmiererin arbeitet. Neue Updates kommen heraus, oder es wird schlichtweg eine neue Entwicklungsumgebung eingesetzt, mit denen die alten Tools nicht mehr verwendet werden können. Aber auch auf personeller Ebene wird sich im Laufe der Zeit eines Softwareprojektes einiges ändern. Neue MitarbeiterInnen kommen hinzu, andere MitarbeiterInnen wechseln in eine andere Abteilung und sind daher nicht mehr für das Projekt verfügbar. Vor allem diese personellen Änderungen führen oft zu großen Problemen, da viele Programmteile von der Person abhängen, die diese implementiert hat. Wenn diese Person jetzt wechselt, muss sich jemand anderes in diesen Programmteil einarbeiten. Dies kann je nach Komplexität sehr stark variieren. Wenn der Ersteller oder die Erstellerin dieses Programmteiles nicht auf lesbaren Code geachtet hat, wird die Zeit für die Einarbeitung noch verlängert. Dies führt dazu, dass viele alte Programmteile nicht mehr geändert werden und bei Fehlern zahlreiche Workarounds implementiert werden. Wenn dieses Problem in einem größeren Kontext betrachtet wird, kann es im schlimmsten Fall zum Scheitern eines Projektes führen, da es unmöglich wird Änderungen vorzunehmen. 

\subsection{Theorie der zerbrochenen Fenster}
\SuperPar Robert C. Martin bringt in seinem Buch \ref{CleanCode} die \textit{Theorie der zerbrochenen Fenster} in Verbindung mit der Softwareentwicklung. Zuerst beschrieben wurde diese Theorie in einem Artikel von James Q. Wilson und George L.Kelling. Die Autoren haben in Ihrem Artikel \ref{JAMESQ.WILSON} unter anderem erwähnt, dass ein zerbrochenes Fenster in eime Gebäude, welches nicht repariert wird, die Zerstörung weiterer Fenster nach sich zieht, da es so aussieht, als ob sich niemand darum kümmern würde. Robert C. Martin verwendet diese Theorie und versucht sie für die Softwareentwicklung anzupassen. Dabei beschreibt er einen sehr ähnlichen Fall wie der eines zerbrochenen Fensters. Bei einem Programmabschnitt, sei es eine Klasse, eine Methode oder ähnliches, welche bereits schlecht gestaltet und unlesbar ist, wird bei zukünftigen Änderungen auch nicht darauf geachtet werden den Code sauber zu gestalten. Es werden immer mehr unsaubere Codeteile hinzugefügt, was schlussendlich von unsauberen Code zu nicht mehr wartbarem Code führen kann. Dieses Problem existiert in allen Teilen der Programmierung. Bei Projekten, bei denen nicht auf einen sauber gestalteten Code geachtet wurde, wird in Zukunft meist auch nicht darauf geachtet.

\section{Lösungsvorschlag}
\SuperPar Um diese Probleme zu vermeiden, sollten Regeln und Grundsätze gefunden werden, wie gut lesbarer bzw. wartbarer Code gestaltet werden kann. Dabei ist es sehr wichtig darauf zu achten, dass der Code meist nicht nur von einer Person bearbeitet wird, sondern von vielen verschiedenen ProgrammiererInnen. Wenn darauf geachtet wird denn Code so zu gestalten, dass er verständlich, gut lesbar und gut strukturiert ist, dann ist es auch nach einigen Wochen, Monaten und Jahren noch gut möglich zu verstehen, was der Zweck der betrachteten Codestelle ist. In den letzten Jahren hat sich zu diesem Thema eine Strömung ergeben welche sich als Clean Code Development (im folgenden nur noch CCD) bezeichnet. Geprägt wurde diese Bewegung in erster Linie von Robert. C. Martin und seinem Bestseller Clean Coder. Umso genauer man darauf achtet, den Code beim Schreiben lesbar zu gestalten, desto leichter wird es diesen auch noch nach einiger Zeit wieder zu Lesen. In dieser Arbeit sollten häufig auftretende Probleme in der Softwareentwicklung betrachtet werden und an Hand der von Robert C. Martin definierten Regelen gezeigt werden, wie diese gelöst werden könnten.

