\section{Klassen}


\subsection{Namensgebung - Klassen}
\begin{itemize}
	\item Betreffende Klasse: \textit{LogLog}
	\item Betreffender Namespace: \textit{log4net.Util}
\end{itemize}

Bei der Namensgebung kommt es immer wieder zu Unklarheiten und Problemen wodurch es oft nötig wird, mit Kommentaren zu beschreiben, wofür diese Komponente verwendet wird. Dabei werden diese Probleme umso größer, je größer der Gültigkeitsbereich dieses Namens ist. Ein Beispiel für eine solche schlechte Namensgebung in einem großen Gültigkeitsbereich in Log4net, ist die in Listing \ref{lst:BadName} dargestellte Klasse \textit{LogLog}. 

\begin{lstlisting}[language={[Sharp]C}, caption=Beispiele für schlechte Namensgebung, label=lst:BadName]
/// <summary>
	/// Outputs log statements from within the log4net assembly.
	/// </summary>
	/// <remarks>
	/// <para>
	/// Log4net components cannot make log4net logging calls.
	/// However, it is sometimes useful for the user to learn 
	/// about what log4net is doing.
	/// </para>
	/// <para>
	/// All log4net internal debug calls go to the standard output 
	/// stream whereas internal error messages are sent to the 
	/// standard error output stream.
	/// </para>
	/// </remarks>
	/// <author>Nicko Cadell</author>
	/// <author>Gert Driesen</author>
	public sealed class LogLog
\end{lstlisting}

Grundsätzlich kann man anhand des Namens \textit{LogLog} keine genauen Aussagen machen, welche Aufgabe diese Klasse erfüllt. Ein Blick in den im Listing \ref{lst:BadName} stehenden Kommentar gibt Aufschluss darüber, dass das Logging über Log4Net für Log4Net Komponenten nicht möglich ist, wodurch es notwendig ist, eine eigene Klasse für das interne Logging zu implementieren. Der Kommentar könnte durch eine bessere Namensgebung für die Klasse überflüssig gemacht werden. Ein Beispiel für einen bessern Namen wäre \textit{Log4NetInternalLogging}, wodurch gleich klar wird, dass diese Klasse nur für das interne Logging zuständig ist. 