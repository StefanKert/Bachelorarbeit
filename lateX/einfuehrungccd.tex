\chapter{Einführung in CCD}
\label{cha:Diplomschrift}

\section{Allgemeines zum CCD}
Wie bereits in im Überblick erläutert, beschäftigt sich CCD in erster Linie mit Praktiken und Strategien, wie man Code so gestalten kann, dass er möglichst leicht zu Lesen, zu Warten und Anzupassen ist. Dabei sind diese Praktiken unabhängig von der Programmiersprache. Auf Grund der Tatsache, dass vor allem ältere Programmiersprachen gewisse Restriktionen wie zum Beispiel eine maximal Länge von Variablen aufweisen sind jedoch nicht alle Praktiken zur Gänze umsetzbar, jedoch kann sich trotzdem daran orientiert werden. Es ist weiterhin nicht relevant welches Programmierparadigma der Programmiersprache zu Grunde liegt. Die meisten Regeln gelten gleichermaßen für objektorientierte, funktionale und imperative Programmiersprachen. Einige Regeln sind auf Grund der unterschiedlichen Eigenschaften der einzelnen Paradigmen nicht zur Gänze umsetzbar, jedoch gibt es auch Abwandlungen der Regeln welche in den unterschiedlichen Paradigmen gelten. 

\section{Warum CCD?}
Die Frage die sich im Zusammenhang mit CCD immer wieder ergibt ist: Warum CCD? Erst in den letzten Jahren hat sich CCD zu einem wirklich wichtigen Teil der Softwareentwicklung entwickelt. Bei einer Suche nach \"Clean Code Development\" ergeben sich 39.600.000 Ergebnisse (Stand November 2015). Auch bei Inseraten für Jobangebote ist Clean Code Development ein sehr wichtiger Punkt: Bei einer Suche nach dem selben Begriff auf stack overflow careers (http://careers.stackoverflow.com) ergeben sich hier 973 Treffer (Stand November 2015).

\SuperPar Diese Zahlen geben Aufschluss darüber, dass ein sehr großes Interesse an CCD besteht und dieses auch in sehr vielen Bereichen Anwendung findet. Dies liegt in erster Linie an einer der 
