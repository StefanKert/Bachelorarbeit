
\section{Hibernate}
\subsection{Allgemeines}
Die Bibliothek Hibernate ist einer der wohl bekannteste OR-Mapper für Java. Mit Hibernate ist es möglich ein objektorientiertes Datenmodell  auf ein relationales Datenbankmodell zu mappen. Die Bibliothek ist mittlerweile über 10 Jahre alt und wurde unter anderem nach .NET portiert und heißt dort NHibernate. Entwickelt wird sie von Red Hat und steht unter der GNU LGPL zur Verfügung. Das Repository lässt sich auf GitHub unter https://github.com/hibernate/hibernate-orm finden.

 
\subsection{Unleserliche boolsche Ausdrücke}
\label{cha:BadBoolStatements}
\begin{itemize}
	\item Betreffende Klasse: \textit{AbstractPropertyHolder}
	\item Betreffendes Paket: \textit{org.hibernate.cfg}
\end{itemize}

\SuperPar Ein weiteres sehr oft auftretendes Problem sind sehr komplexe boolsche Ausdrücke. Oft wird auf Grund ihrer Komplexität ein Kommentar hinzugefügt, der Aufschluss darüber geben sollte, was genau gewünscht ist. Jedoch ist die Komplexität meist nicht das einzige Problem. Hierzu ist der in Listing \ref{BoolStatement1} abgebildete Code aus der Klasse \textit{AbstractPropertyHolder} ein sehr gutes Beispiel.

\begin{lstlisting}[language=Java, caption=Komplexe boolsche Ausdrücke 1 Zeile 255 - 264, label=lst:BoolStatement1]
	@Override
	public JoinColumn[] getOverriddenJoinColumn(String propertyName) {
		JoinColumn[] result = getExactOverriddenJoinColumn( propertyName );
		if ( result == null && propertyName.contains( ".collection&&element." ) ) {
			//support for non map collections where no prefix is needed
			//TODO cache the underlying regexp
			result = getExactOverriddenJoinColumn( propertyName.replace( ".collection&&element.", "."  ) );
		}
		return result;
	}
\end{lstlisting}

Ein paar Zeilen weiter in derselben Klasse findet sich die in Listing \ref{lst:BoolStatement2} dargestellte Methode.

\begin{lstlisting}[language=Java, caption=Komplexe boolsche Ausdrücke 2 Zeile 305 - 313, label=lst:BoolStatement2]
public JoinTable getOverriddenJoinTable(String propertyName) {
	JoinTable result = getExactOverriddenJoinTable(propertyName);
	if(result == null 
		&& propertyName.contains(".collection&&element.")){
		//support for non map collections where no prefix is needed
		//TODO cache the underlying regexp
		result = getExactOverriddenJoinTable( propertyName.replace( ".collection&&element.", "."  ) );
	}
	return result;
}
\end{lstlisting}

Konkret geht es in diesem Beispiel um den in Listing \ref{lst:BoolStatement3} beschriebenen boolschen Ausdruck der in dieser Klasse insgesamt drei mal vorkommt.

\begin{lstlisting}[language=Java, caption=Boolscher Audruck, label=lst:BoolStatement3]
result == null && propertyName.contains(".collection&&element.")
\end{lstlisting}

Wie man anhand dieser Listings leicht erkennen kann, wird dieser Ausdruck in beiden Methoden verwendet, wodurch sich eine Code Duplizierung ergibt. Ein weiteres Problem, das sich hier wie bereits oben beschrieben ergibt, ist die Tatsache, dass dieser boolsche Kommentar ohne einen Kommentar oder ein genaues Lesen nicht sehr leicht zu verstehen ist. Man könnte hier beide Probleme beseitigen, in dem man aus dem boolschen Audruck eine eigene Methode extrahiert. Dabei wird hier nur der zweite Teil des boolschen Ausdrucks extrahiert, da der erste Teil mit dem Nullcheck spezifisch für die einzelnen Methoden verwendet werden muss. Eine solche Methode könnte wie in Listing \ref{BoolStatement4} aussehen.

\begin{lstlisting}[language=Java, caption=Boolscher Ausdruck neu, label=lst:BoolStatement4]
private bool isPlaceholderForCollectionAndElementInPropertyName(String propertyName) {
	return propertyName.contains(".collection&&element.");
}
\end{lstlisting}

Durch ein Extrahieren der Überprüfung des Parameters \textit{propertyName} kann für diese Methode ein eigener Name gewählt werden, der klar den Zweck dieser Überprüfung klar machen sollte. Durch diese eigene Methode wird außerdem die Codeduplizierungen eliminiert und es bleibt nur noch die Sicherheitsüberprüfung auf null.

\subsection{Fazit}
Auch beim Hibernate Framework merkt man, dass es bereits seit einigen Jahren entwickelt wird. Im Grunde genommen wirkt es sehr ausgereift und auch sehr durchdacht. Einige Dinge, wie überflüssige Kommentare, auskommentierter Codeabschnitte, aber auch Codeduplizierungen könnten vermieden werden, in dem man kleinere Refactorings vornimmt. Was besonders positiv auffällt ist, dass selbst die Testsuite sehr sauber gehalten wurde und das auch dort auf Clean Code Richtlinien geachtet wurde. 
