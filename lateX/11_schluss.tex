\chapter{Schlussbemerkungen}
\label{cha:Schluss}
In diesem Abschnitt möchte ich dem Leser noch einen persönliche Meinung, sowie einen Einblick in den Prozess CCD in einem Unternehmen einzuführen geben.
\section{Fazit zu CCD}
 Grundsätzlich bin ich der Meinung, dass CCD einen wirklichen Mehrwert für alle Stakeholder einer Software bzw. des Unternehmens welches die Software entwickelt bietet. Durch das Achtgeben auf die Regeln, welche Robert C. Martin in seinem Buch sehr schön definiert ist es auch für Anfänger in der Programmierung sehr leicht möglich besseren Code zu schreiben. Wie auch Robert C. Martin bin ich der Meinung, dass man nicht von heute auf morgen Clean Code programmieren kann und dass selbst erfahrene Programmierer es nicht schaffen gleich auf Anhieb die richtige Lösung, welche gut verständlich, sauber und leicht zu warten ist zu finden. Daher ist es meiner Meinung nach umso wichtiger, dass man die \textit{Pfadfinderregel} beachtet und Code denn man liest immer ein bisschen verbessert. Wie bereits erwähnt bringt CCD, wenn es bewusst betrieben wird, einen Mehrwert für alle Stakeholder. 
\begin{itemize}
	\item Kunde: Wenn der Kunde neue Features benötigt können diese schneller umgesetzt werden
	\item Unternehmen: Durch die saubere Gestaltung des Codes Kosten Änderungen weniger
	\item Projektleiter: Die Zufriedenheit der Mitglieder eines Entwicklungsteams ist sehr viel höher, wenn der Code leicht lesbar ist
	\item Entwickler: Durch bessern Code können Änderungen leichter vorgenommen werden und es ist meist nicht nötig stundenlang zu Debuggen bis herausgefunden werden kann welchen Zweck die betrachtete Funktion erfüllt
\end{itemize}


\section{Persönliche Erfahrungen mit CCD}
Ich persönlich praktiziere seit mehreren Jahren CCD und wir haben auch in der Firma einige Schulungen absolviert um unseren Codierungsstil zu verbessern. Dadurch haben sich für uns die von Robert C. Martin prognostizierten Vorteile ergeben. Wir können schneller und leichter Fehler beheben, neue Features hinzufügen und der Code ist auch noch nach Wochen und Monaten gut lesbar. Dabei greifen wir intern vor allem auf das abhalten von Code Reviews zurück, bei denen wir gegenseitig die geschriebenen Codeabschnitte betrachten und verbessern. Dabei hat sich für uns eine Kultur eingestellt, dass wir in allen Projekten einen durchgängig sauberen Codierungsstil haben. Probleme gab es beim Einführen von CCD, da das Management nicht wirklich überzeugt davon war, dass wir Vorteile daraus haben den Code sauberer zu gestalten, da davon ausgegangen wurde, dass die Implementierung dadurch viel länger benötigen wird. Es wurde dann schrittweise vorgegangen und durch den Erfolg, welchen wir mit den ergriffenen Maßnahmen hatten wurde es uns ermöglicht noch weitere Maßnahmen einzuführen. Da die von uns entwickelte Software bereits mehrere Jahre alt ist, müssen auch zahlreiche Komponenten neu geschrieben werden. Vor allem bei solchen Refakturierungen fallen die Schwierigkeiten auf, die auftreten, wenn nicht darauf geachtet wurde Clean Code zu schreiben. Variablennamen sind teilweise sehr schlecht gewählt und es kann daher nicht genau gesagt werden wofür diese steht, wodurch wiederum ein längeres Einlesen in den Code und im schlimmsten Fall sogar ein Debuggen notwendig wird. Durch eine sukzessive Neuimplementierung dieser Komponenten unter Beachtung der CCD Kriterien können wir auch einen Benefit für den Kunden liefern, da wir leichter Erweiterungen implementieren können, ohne dass wir das System beeinflussen. Ein sehr wichtiger Punkt in dieser Hinsicht, denn wir auch bei der Entwicklung verbessern müssen ist die Testabdeckung. 


