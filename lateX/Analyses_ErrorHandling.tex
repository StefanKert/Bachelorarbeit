\section{Fehlerbehandlung}



\subsection{Kein Null zurückgeben}
\begin{itemize}
	\item Projekt: \textit{CoreFx}
	\item Programmiersprache: \textit{C\#}
	\item Betreffende Klasse: \textit{ProcessModuleCollection}
	\item Betreffender Namespace: \textit{System.Diagnostics}
\end{itemize}

\SuperPar In modernen Programmierumgebungen wie Java und C\# stellen die sogenannten \textit{Nullreferenzausnahmen} eine der häufigsten Ausnahmefälle dar. Diese führen dazu, dass beim Zugriff auf ein Objekt, welches den Wert \textit{null} besitzt, ein Laufzeitfehler auftritt, welcher behandelt werden muss. Noch problematischer sind diese Ausnahmen im C++ - Bereich, da dort keine Ausnahme ausgelöst wird, falls das Objekt den Wert null besitzt, da jedes Objekt auf einen gewissen Speicherbereich verweist und dieser Speicherbereich aber vorhanden ist. In C++ führt der Zugriff auf ein \textit{Null-Objekt} zu einem Zugriff auf einen ungültigen Speicherbereich. Solche Fehler sind sehr schwer nachzuvollziehen und führen häufig zu großen Problemen wenn die Software bereits im Betrieb ist. Meist wird diesen Problem durch zahlreiche Überprüfungen ob der zurückgegebene Wert \textit{null} ist, vorgebeugt. Dies führt aber zu stark überladenen Methoden und kann es kann auch sehr schnell darauf vergessen werden diese Überprüfungen einzubauen.

\SuperPar Diese Probleme und die Überladung des Codes mit Überprüfungen können durch das Verzichten auf die Rückgabe von Null Werten verhindert werden. Für Enumerationstypen wie Listen oder ähnlichem sollte eine leere Liste zurückgegeben werden. Meist wird für Listen nach dem Aufruf eine Form der Iteration durchgeführt. Entweder wird in einer \textit{for-Schleife} über die einzelnen Elemente der Liste iteriert, was bei einer leeren Liste einfach dazu führt, dass die Schleife nicht durchlaufen wird. Bei Objekten gibt es die Möglichkeit das sogenannte \textit{Nullobjectpattern} zu verwenden. Bei diesem Pattern wird statt des Wertes \textit{null} eine leere Implementierung des Objektes zurückgegeben. Ein Beispiel, wo ein solches \textit{Nullobjectpattern} Anwendung finden könnte wäre der Codeabschnitt in Listing \ref{lst:ReturnZero}. 

\begin{lstlisting}[language={[Sharp]C}, caption=Beispiele für Rückgabe eines Null Wertes, label=lst:ReturnZero]
protected List<ProcessModule> InnerList
{
	get
	{
			if (_list == null)
				_list = new List<ProcessModule>();
		return _list;
	}
}
\end{lstlisting}

\SuperPar Hier wird zu erst intern überprüft, ob die Liste, welche in einer Membervariable gespeichert ist \textit{null} ist, wenn ja wird der Wert dieser auf eine leere Liste gesetzt und somit können \textit{Nullreferenzausnahmen} verhindert werden. 


